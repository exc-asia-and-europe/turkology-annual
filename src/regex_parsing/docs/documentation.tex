%%This is a very basic article template.
%%There is just one section and two subsections.
\documentclass{scrartcl}
\usepackage[utf8]{inputenc}

\usepackage{covington}

\title{Turkology Annual Online}
\subtitle{Technical Documentation}
\author{Matthias Arnold, Nicolas Bellm, Arina Chitavong, Mateusz Dolata,
\\Anette Frank, Peter Gietz, Jens Hansche, Dustin Heckmann, \\Christian Roth,
Michael Ursinus}

\begin{document}

\maketitle
\section{Introduction}

\section{Architecture}
The architecture is split into three big parts. The first part is the scanning
and the OCR with ABBYY FineReader 9.0 Corporate Edition, the second part is the
parsing and the storing into a PostgreSQL database and the last part is the web
frontend for acessing the database.

First we scanned the printed pages of the 26 volumes of the Turkology Annual.
Then we used the Optical Character Recognition (OCR) software ABBYY FineReader
9.0 Corporate Edition to recoginze the scanned text. Therefore we used about 20
scanning languages. Unfortunately, there was no output format which presevered
all inforamtion which was gathered by the OCR software. So we used two different
output formats to process the scanning output further. The first output format
was WordML. This output format preserves hyphenation, formatting and the
languages. The second output format was PDF with the original image as overlay.
Thus we found out the page number and the exact position in the page of a piece
of text. This PDF was converted to plain text with a special patched version of
pdftotext which is a part of xpdf to preserve the position information.

For the next step we used the programming language Python 3. In this step we
aligned the converted text file and the WordML file. Then we parsed the
structure of the entries in the Turkology Annual with the help of LEPL 3, a
recursive decent parser for Python 3. We stored the parsing results in an
intermediate Python format. The access to the PostgreSQL database happened with
SQLAlchemy, a Python SQL Toolkit and Object Relational Mapper (ORM). From the
intermediate format we stored the parsed data into this database.

The webfront end is based on the web framework Django. As search engine we used
PyLucene which is based on Apache Lucene. So we was able to use the same Object
Relational Mapping as we used for the database access in the second step.



\section{Difficulties}
Our approach based on hand-craftet rules and multi-level architecture was not 
successful with regard to a number of entries from the Turkology
Annual. Even though, we were controlling the precision of the developed rules
throughout the course of the project, and we tried to improve their performance,
we did not reach a point were all the records extracted from the scanned data
were correctly parsed. Particularly, introducing general patterns
resulted in the increase of false positive, whereas using more precise ones
leaded to the decrease of true positives and required the development of further
rules with ascending complexity. 

In the following, we present several examples of incomplete or incorrect parsing.
In order to keep our investigation organised we introduce several \emph{ad-hoc}
groups corresponding to the reason for failed analysis. The presented examples
are assumed to be representative for the majority of incomplete records within
the database. We do not, however, aim to present a full topology of them, i.e.,
the introduced groups do not need to embrace all incorrect parses. For
convenience reasons, we will offer the correct interpretation and the record
from the database for every investigated case (cf. Appendix 1).

\subsection{OCR Problems}
This section presents a handful examples of incorrect entries' parsing arising
from the inaccuracy of \emph{optical character recognition}. Basically, the
multilanguage OCR used within the project does not yield many errors in the
recognition of word-like string sequences. Furthermore, such errors does not
have much influence on the parsing, so that the database's records are mostly
complete in such cases. Conversely, there were lots of mistakes regarding the
punctuation marks, which are of great importance to our rule-based paradigm.
Most of the entries containing such mistakes could be parsed only partially or
not at all. 

\subsubsection{Punctuation}

There are lots of problems resulting from bad recognition of punctuation marks. 

\begin{example}
\begin{verbatim}
1303. Fisher, Alan W. The Ottoman Crimea in the mid-seventeenth 
century. Some problems and preliminary considerations. In: 
TA 9.193.215^226.
\end{verbatim}
\label{e1}
\end{example}
 
In this example the dash between \verb+215+ and \verb+226+ was recognised by
OCR-Machinery as \verb+^+. Therefore the record in the database looks like
presented in Figure \ref{e1:f1}, page \pageref{e1:f1}. It should, however, look like
Figure \ref{e1:f2}, page \pageref{e1:f2}.

\subsubsection{Numbers}
It happens also very often, that the numbers (especially in references) are read
as other characters. This is so in the example (\ref{e2}), where \verb+1+ has
been recognised as \verb+^+, what again resulted in non-complete record in the
database. Moreover, the case presented included a confusing year format and a
space before \verb+361-376+, what could also lead to the incorrect parse (cf.
Figure \ref{e2:f1} and Figure \ref{e2:f2}, page \pageref{e2:f1}).

\begin{example}
\begin{verbatim}
846. Basar, Haşmet Mustafa Kemal Atatürk's thoughts on rural 
development and co-operative movement. In:
ÎFM 38.1^.1980-1981(1982). 361-376.
\end{verbatim}
\label{e2}
\end{example}

\subsection{Complicated TA-Notation}
Another kind of errors are those resulting from confusing, or simply complicated
notation used within the Turkology Annual. In most cases errors were caused by
unforeseeable specifications of references or comments. 

\subsubsection{References}
The syntax of references (both intern and extern) is sometimes difficult to
understand and to foresee. An interesting example is included in (\ref{e3}). The
resulting records can be seen on page \pageref{e3:f1}.

\begin{example}
\begin{verbatim}
745. Delìgönül, Ethem Millî Mücadelede eski Tophaneliler. In: MiKü
3.1.1981.32-35, 2.42-45, 3.44-49. [Die Absolventen der 
Militärgewerbeschule (Askerî san'at mektebi in Tophane, istanbul) 
im nationalen Unabhängigkeitskrieg . ] 
\end{verbatim}
\label{e3} 
\end{example}

The example presented below includes a kind of double reference, whereas the
second one, introduced by \verb+Auch erschienen in+ is rather an exception than
the rule in the notation of Turkology Annual. The algorithm developed in the
projec was able to extract the title and the comment, but the reference could
not be resolved (c.f. Figures \ref{e4:f1} and \ref{e4:f2}, page \pageref{e4:f1})

\begin{example}
\begin{verbatim}
295. Ivanova, Marija Dumata ile v sistemata na turskite sledlozi, 
nejnite funkcii i săotvetstvijata ì v bălgarskija ezik. In: 
BE 31.5.1981.448-451. Auch erschienen in: SåpE 1981.1.54-60. [Das
Wort ile im System der türkischen Postpositionen, seine Funktionen
und seine Entsprechungen im Bulgarischen.]
\end{verbatim}
\label{e4} 
\end{example}

\subsubsection{Comments}
Some entries could not be parsed properly because of the exceptional structure
of comments, which normally should be included between $[$ and $]$. The
parentheses can be followed by a sequence of characters. But sometimes even this
flexible pattern did not capture entries from TA, as the example (\ref{e5}) and
the corresponding record (Figure \ref{e5:f1} and \ref{e5:f2}, page \pageref{e5:f1})
show.

\begin{example}
\begin{verbatim}
263. Türkiye'de halk ağzından derleme sözlüğü 
[s. TA 1.203,2.211,5.243]. Weitere Bände: Bd. 10: S-T. Ankara,
1978, S. 3507-4018 (TDK 211/10), Bd. 11: U-Z. Ankara 1979, S. 
4019^402 (TDK 211/11), Bd. 12: Ek 1. Ankara, 1982, S. XIV+
4403-4842(TDK 211/12).
\end{verbatim}
\label{e5} 
\end{example}

\subsection{Parsing}
Some of the records within the database are not parsed properly because of the
rule-based paradigm underlaying our parsing algorithm. In particular, we try to
match the entries from TA to a number of patterns, from the specific ones to the
more general ones. Sometimes, the input string is not matched by any of the
patterns. In some other cases it is not matched by the appropriate specific
pattern, but it is captured by a general one. Both cases are discussed below.

\subsubsection{Too general parse}
Example (\ref{e6}) presents a formally properly constructed entry for a
collection including the information on editors (\emph{Richard L. Chambers—Günay
Kut (Alpay) ed.}). The latter does not, however, exhibit a standard form.
Firstly, it includes two editors. Secondly, the name of the first person includes the
initial of the middle name followed by a period, which is at the same time a
quite important idiosyncratic symbol, terminating, e.g. titles or comments. And,
last but not least, this string includes parentheses, which is also unusual. For the
resulting record, cf., Figure \ref{e6:f1} and \ref{e6:f2}, page \pageref{e6:f1}.

\begin{example}
\begin{verbatim}
316. Contemporary Turkish short stories. An intermediate reader.
Richard L. Chambers—Günay Kut (Alpay) ed. Minneapolis—Chicago, 
1977, XI + 159 S. (Middle Eastern Languages and Linguistics, 3). 
\end{verbatim}
\label{e6} 
\end{example}

\subsubsection{No parse possible}
Some non-standard entries resulted in no parse at all. This can have several
reasons, and mostly, for every case special pattern would be needed. Another
approach could be based on statistical methods, that use the properly
recognised entries as training set. We assume, that such an approach would
significantly decrease the number of non-parsed recods.
The example below includes an entry that could not be parsed properly by our
algorithm. It is so probably due to the many numbers occurring in the title. And
due to the small mistake in the reference (there is a space between \verb+1985.+ and
\verb+113-130.+). We present the proper parse for this example in Figure
\ref{e7:f2}, page \pageref{e7:f2}.

\begin{example}
\begin{verbatim}
586. Gallotta, Aldo Venise et l'Empire ottoman, de la paix du 25
janvier 1479 à la mort de Mahomet II, 3 mai 1481. 
In: ROMM 39.1985. 113-130.
\end{verbatim}
\label{e7} 
\end{example}


\section{Conclusions} 

\section*{Appendix 1}

\begin{figure}[ht]
\begin{tabular}{|p{0.14\textwidth}p{0.8\textwidth}|}
ID:&  	20924\\
Type:& 	article\\
Autor:& 	Fisher, Alan W.\\
Title:& 	The Ottoman Crimea in the mid-seventeenth century. Some problems and
preliminary considerations.
\end{tabular}
\caption{Database record for the example in (\ref{e1}) \label{e1:f1}}
\end{figure} 

\begin{figure}[ht]
\begin{tabular}{|p{0.14\textwidth}p{0.8\textwidth}|}
ID:&  	20924\\
Type:& 	article\\
Autor:& 	Fisher, Alan W.\\
Title:& 	The Ottoman Crimea in the mid-seventeenth century. Some problems and
preliminary considerations.\\
In:& TA 9.193.215-226. 	
\end{tabular}
\caption{Correct record for the example in (\ref{e1}) \label{e1:f2}}
\end{figure}

\begin{figure}[ht]
\begin{tabular}{|p{0.14\textwidth}p{0.8\textwidth}|}
ID:  	&20227\\
Type: 	&article\\
Autor: 	&Basar, Haşmet\\
Title: 	&Mustafa Kemal Atatürk's thoughts on rural development and co-operative
movement.
\end{tabular}
\caption{Database record for the example in (\ref{e2}) \label{e2:f1}}
\end{figure} 

\begin{figure}[ht]
\begin{tabular}{|p{0.14\textwidth}p{0.8\textwidth}|}
ID:  	&20227\\
Type: 	&article\\
Autor: 	&Basar, Haşmet\\
Title: 	&Mustafa Kemal Atatürk's thoughts on rural development and
co-operative\\
In: 	&ÎFM 38.11.1980-1981(1982).361-376.
\end{tabular}
\caption{Correct record for the example in (\ref{e2}) \label{e2:f2}}
\end{figure}

\begin{figure}[ht]
\begin{tabular}{|p{0.14\textwidth}p{0.8\textwidth}|}
ID:  	&20106 \\
Type: 	&article \\
Autor: 	&Delìgönül, Ethem \\
Title: 	&Millî Mücadelede eski Tophaneliler.	\\
Comment: 	&$[$Die Absolventen der Militärgewerbeschule (Askerî san'at mektebi in
Tophane, istanbul) im nationalen Unabhängigkeitskrieg .$]$ 
\end{tabular}
\caption{Database record for the example in (\ref{e3}) \label{e3:f1}}
\end{figure} 

\begin{figure}[ht]
\begin{tabular}{|p{0.14\textwidth}p{0.8\textwidth}|}
ID:  	&20106 \\
Type: 	&article \\
Autor: 	&Delìgönül, Ethem \\
Title: 	&Millî Mücadelede eski Tophaneliler.	\\ 
Comment: 	&$[$Die Absolventen der Militärgewerbeschule (Askerî san'at mektebi in
Tophane, istanbul) im nationalen Unabhängigkeitskrieg .$]$  \\ 
In: &MiKü 3.1.1981.32-35, 2.42-45, 3.44-49
\end{tabular}
\caption{Correct record for the example in (\ref{e3}) \label{e3:f2}}
\end{figure}

\begin{figure}[ht]
\begin{tabular}{|p{0.14\textwidth}p{0.8\textwidth}|}
ID:  	&19549 \\
Type: 	&article \\
Autor: 	&Ivanova, Marija \\
Title: 	&Dumata ile v sistemata na turskite sledlozi, nejnite funkcii i
săotvetstvijata ì v bălgarskija ezik.  \\
Comment:& 	$[$Das Wort ile im System der türkischen Postpositionen, seine
Funktionen und seine Entsprechungen im Bulgarischen.$]$
\end{tabular}
\caption{Database record for the example in (\ref{e4}) \label{e4:f1}}
\end{figure} 

\begin{figure}[ht]
\begin{tabular}{|p{0.14\textwidth}p{0.8\textwidth}|}
ID:  	&19549 \\
Type: 	&article \\
Autor: 	&Ivanova, Marija \\
Title: 	&Dumata ile v sistemata na turskite sledlozi, nejnite funkcii i
săotvetstvijata ì v bălgarskija ezik.  \\
Comment:& 	$[$Das Wort ile im System der türkischen Postpositionen, seine
Funktionen und seine Entsprechungen im Bulgarischen.$]$ \\
In:		& BE 31.5.1981.448-451.; SåpE 1981.1.54-60.
\end{tabular}
\caption{Correct record for the example in (\ref{e4}) \label{e4:f2}}
\end{figure}

\begin{figure}[ht]
\begin{tabular}{|p{0.14\textwidth}p{0.8\textwidth}|}
ID:  	&19509 \\
Type: 	&collection \\
Title: 	&Türkiye'de halk ağzından derleme sözlüğü $[$s. TA 1.203,2.211,5.243$]$.
Weitere Bände: Bd. 10: S-T. \\
Location: &	Ankara \\
Year: 	&1978 \\
Comment:& 	4019\verb+^+402 (TDK 211/11), Bd. 12: Ek 1. Ankara, 1982, S. XIV+
4403-4842(TDK 211/12).
\end{tabular}
\caption{Database record for the example in (\ref{e5}) \label{e5:f1}}
\end{figure} 

\begin{figure}[ht]
\begin{tabular}{|p{0.14\textwidth}p{0.8\textwidth}|}
ID:  	&19509 \\
Type: 	&collection \\
Title: 	&Türkiye'de halk ağzından derleme sözlüğü  \\
Location: &	Ankara \\
Year: 	&1978 \\
Comment:& 	$[$s. TA 1.203,2.211,5.243$]$.
Weitere Bände: Bd. 10: S-T. 4019\verb+^+402 (TDK 211/11), Bd. 12: Ek 1. Ankara,
1982, S. XIV+ 4403-4842(TDK 211/12).
\end{tabular}
\caption{Correct record for the example in (\ref{e5}) \label{e5:f2}}
\end{figure}


\begin{figure}[ht]
\begin{tabular}{|p{0.14\textwidth}p{0.8\textwidth}|}
ID:&  	5897\\
Type:& 	collection \\
Title:& 	Contemporary Turkish short stories. An intermediate reader. Richard L.
Chambers—Günay Kut (Alpay) ed. \\
Location:& 	Minneapolis—Chicago \\
Year:& 	1977 \\
Comment:& 	(Middle Eastern Languages and Linguistics, 3).
\end{tabular}
\caption{Database record for the example in (\ref{e6}) \label{e6:f1}}
\end{figure} 

\begin{figure}[ht]
\begin{tabular}{|p{0.14\textwidth}p{0.8\textwidth}|}
ID:&  	5897\\
Type:& 	collection \\
Title:& 	Contemporary Turkish short stories. An intermediate reader.  \\
Location:& 	Minneapolis—Chicago \\
Year:& 	1977 \\
Comment:& 	(Middle Eastern Languages and Linguistics, 3).\\
Editor:& Richard L. Chambers; Günay Kut (Alpay)
\end{tabular}
\caption{Correct record for the example in (\ref{e6}) \label{e6:f2}}
\end{figure}

\begin{figure}[ht]
\begin{tabular}{|p{0.14\textwidth}p{0.8\textwidth}|}
ID:  	&38742 \\
Type: 	&article \\
Autoren: &Galotta, Aldo \\ 	
Title: 	&Venise et l'Empire ottoman, de la paix du 25 janvier 1479 à la mort de
Mahomet II, 3 mai 1481. \\
In: 	&ROMM 39.1985. 113-130.
\end{tabular}
\caption{Correct record for the example in (\ref{e7}) \label{e7:f2}}
\end{figure}

\end{document}
 
