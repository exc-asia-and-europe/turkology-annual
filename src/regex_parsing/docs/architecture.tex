The architecture is split into three big parts. The first part is the scanning and the OCR with ABBYY FineReader 9.0 Corporate Edition, the second part is the parsing and the storing into a PostgreSQL database and the last part is the web frontend for acessing the database.

First we scanned the printed pages of the 26 volumes of the Turkology Annual. Then we used the Optical Character Recognition (OCR) software ABBY FineReader 9.0 Corporate Edition to recoginze the scanned text. Therefore we used about 20 scanning languages. Unfortunately, there was no output format which presevered all inforamtion which was gathered by the OCR software. So we used two different output formats to process the scanning output further. The first output format was WordML. This output format preserves hyphenation, formatting and the languages. The second output format was PDF with the original image as overlay. Thus we found out the page number and the exact position in the page of a piece of text. This PDF was converted to plain text with a special patched version of pdftotext which is a part of xpdf to preserve the position information.

For the next step we used the programming language Python 3. In this step we aligned the converted text file and the WordML file. Then we parsed the structure of the entries in the Turkology Annual with the help of LEPL 3, a recursive decent parser for Python 3. We stored the parsing results in an intermediate Python format. The access to the PostgreSQL database happened with SQLAlchemy, a Python SQL Toolkit and Object Relational Mapper (ORM). From the intermediate format we stored the parsed data into this database.

The webfront end is based on the web framework Django. As search engine we used PyLucene which is based on Apache Lucene. So we was able to use the same Object Relational Mapping as we used for the database access in the second step.
